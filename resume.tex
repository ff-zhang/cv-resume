%-------------------------
% Resume in Latex
% Author : Jake Gutierrez
% Based off of: https://github.com/sb2nov/resume
% License : MIT
%------------------------

\documentclass[letterpaper,11pt]{article}

\usepackage{latexsym}
\usepackage[empty]{fullpage}
\usepackage{titlesec}
\usepackage{marvosym}
\usepackage[usenames,dvipsnames]{color}
\usepackage{verbatim}
\usepackage{enumitem}
\usepackage[hidelinks]{hyperref}
\usepackage{fancyhdr}
\usepackage[english]{babel}
\usepackage{tabularx}
\usepackage{fontawesome5}
\input{glyphtounicode}

\usepackage{amsmath, amsfonts, amsthm}
\usepackage{censor}

%----------FONT OPTIONS----------
% sans-serif
% \usepackage[sfdefault]{FiraSans}
% \usepackage[sfdefault]{roboto}
% \usepackage[sfdefault]{noto-sans}
% \usepackage[default]{sourcesanspro}

% serif
% \usepackage{CormorantGaramond}
% \usepackage{charter}


\pagestyle{fancy}
\fancyhf{} % clear all header and footer fields
\fancyfoot{}
\renewcommand{\headrulewidth}{0pt}
\renewcommand{\footrulewidth}{0pt}

% Adjust margins
\addtolength{\oddsidemargin}{-0.5in}
\addtolength{\evensidemargin}{-0.5in}
\addtolength{\textwidth}{1in}
\addtolength{\topmargin}{-.5in}
\addtolength{\textheight}{1.0in}

\urlstyle{same}

\raggedbottom
\raggedright
\setlength{\tabcolsep}{0in}

% Sections formatting
\titleformat{\section}{
  \vspace{-4pt}\scshape\raggedright\large
}{}{0em}{}[\color{black}\titlerule \vspace{-5pt}]

% Ensure that generate pdf is machine readable/ATS parsable
\pdfgentounicode=1

%-------------------------
% Custom commands
\newcommand{\resumeItem}[1]{
  \item\small{
    {#1 \vspace{-2pt}}
  }
}

\newcommand{\resumeSubheading}[4]{
  \vspace{-2pt}\item
    \begin{tabular*}{0.97\textwidth}[t]{l@{\extracolsep{\fill}}r}
      \textbf{#1} & #2 \\
      \textit{\small#3} & \textit{\small #4} \\
    \end{tabular*}\vspace{-7pt}
}

\newcommand{\resumeSubheadingTwo}[2]{
  \vspace{-2pt}\item
    \begin{tabular*}{0.97\textwidth}[t]{l@{\extracolsep{\fill}}r}
      \textbf{#1} & #2
    \end{tabular*}\vspace{-7pt}
}

\newcommand{\resumeSubSubheading}[2]{
    \item
    \begin{tabular*}{0.97\textwidth}{l@{\extracolsep{\fill}}r}
      \textit{\small#1} & \textit{\small #2} \\
    \end{tabular*}\vspace{-7pt}
}

\newcommand{\resumeProjectHeading}[2]{
    \item
    \begin{tabular*}{0.97\textwidth}{l@{\extracolsep{\fill}}r}
      \small#1 & #2 \\
    \end{tabular*}\vspace{-7pt}
}

\newcommand{\resumeSubItem}[1]{\resumeItem{#1}\vspace{-4pt}}

\renewcommand\labelitemii{$\vcenter{\hbox{\tiny$\bullet$}}$}

\newcommand{\resumeSubHeadingListStart}{\begin{itemize}[leftmargin=0.15in, label={}]}
\newcommand{\resumeSubHeadingListEnd}{\end{itemize}}
\newcommand{\resumeItemListStart}{\begin{itemize}}
\newcommand{\resumeItemListEnd}{\end{itemize}\vspace{-5pt}}

\newcommand{\rom}[1]{\uppercase\expandafter{\romannumeral #1\relax}}

%-------------------------------------------
%%%%%%  RESUME STARTS HERE  %%%%%%%%%%%%%%%%%%%%%%%%%%%%

\begin{document}

\StopCensoring

%----------HEADING----------
% \begin{tabular*}{\textwidth}{l@{\extracolsep{\fill}}r}
%   \textbf{\href{http://sourabhbajaj.com/}{\Large Sourabh Bajaj}} & Email : \href{mailto:sourabh@sourabhbajaj.com}{sourabh@sourabhbajaj.com}\\
%   \href{http://sourabhbajaj.com/}{http://www.sourabhbajaj.com} & Mobile : +1-123-456-7890 \\
% \end{tabular*}

\begin{center}
    \textbf{\LARGE \scshape \censor{Felix Zhang}} \\ \vspace{1pt}
    \footnotesize\faIcon{linkedin} \small\censor{\href{https://www.linkedin.com/in/felixfzhang/}{\underline{felixfzhang}}} $|$ 
    \footnotesize\faIcon{github} \small\censor{\href{https://github.com/ff-zhang}{\underline{ff-zhang}}} $|$ 
    \footnotesize\faIcon{envelope} \censor{\href{mailto:felixf.zhang@utoronto.ca}{\underline{felixf.zhang@utoronto.ca}}}
    % \footnotesize\faIcon{phone} \censor{(778) 678-7918}
    % \footnotesize\faIcon{globe} \censor{\href{https://ff-zhang.github.io/}{\underline{ff-zhang.github.io}}}
\end{center}


%-----------EDUCATION-----------
\section{Education}
  \resumeSubHeadingListStart
    \resumeSubheading
      {University of Toronto}{Sept. 2021 -- May 2025}
      {Honours Bachelor of Science; Specialist in Computer Science, Major in Mathematics}{\censor{3.95}/4.0 cGPA}
      \resumeItemListStart
      \resumeItem{\textbf{Coursework:}
      % Software Design; 
      % Software Tools and Systems;
      Operating Systems;
      % Intro.\ to AI; % Artificial Intelligence; 
      % Intro.\ to Machine Learning;
      Neural Networks and Deep Learning;
      Probabilistic Learning and Reasoning;
      Intro.\ to Image Understanding;
      % Enriched Data Structures and Analysis; 
      Algorithm Design, Analysis and Complexity; 
      Advanced Algorithm Design;
      % Computer Organization; 
      Computational Complexity and Computability;
      % Computability and Logic;
      % Analysis \rom{2};
      Complex Analysis \rom{1};
      Real Analysis \rom{1};
      % Groups, Rings, and Fields;
      Mathematical Probability
      % Algebra \rom{2};
      % Probability and Statistics \rom{1}
      \unskip}
      \resumeItem{\textbf{Awards:}
      University of Toronto Scholar (\$\censor{7 500}),
      Malcolm Wallace Scholarship (\$\censor{4 500}),
      Louis Savlov Scholarships in Sciences And Humanities (\$\censor{500}),
      Dean's List Scholar
      % BC Achievement Scholarship (\$\censor{1 250}),
      % District/Authority Scholarship (\$\censor{1 250})
      }
      \resumeItemListEnd
  \resumeSubHeadingListEnd

%-----------PROGRAMMING SKILLS-----------
\section{Technical Skills}
  \begin{itemize}[leftmargin=0.15in, label={}]
      \item\begin{tabular}{l@{\hspace{0.8em}}l}
      \small\textbf{Languages} & \small{C, C++, Python, Rust, Bash, Java} \\
      \small\textbf{Frameworks} & \small{PyTorch, TensorFlow, scikit-learn, NumPy, Pandas, SciPy, Matplotlib, MuJoCo, Eigen3} \\ % Vue.js, Django, Bootstrap4, Redis, MySQL, spaCy
      \small\textbf{Tools} & \small{Git, Google Colab, gdb, Linux, shell, Jupyter, Anaconda, WSL, Slurm, CMake}
    \end{tabular}
  \end{itemize}

%-----------EXPERIENCE-----------
  \section{Experience}
    \resumeSubHeadingListStart
    \resumeSubheading
    {\small{Machine Learning Researcher} \href{https://github.com/ff-zhang/mocap-mujoco}{\footnotesize\faIcon{external-link-alt}}}{\small{July 2023 -- Present}}
    {\href{https://bmolab.artsci.utoronto.ca/}{BMO Lab}, University of Toronto}{}
      \resumeItemListStart
      \resumeItem{Worked under Prof.\ David Rokeby to integrate motion-capture suits and diffusion models into live performances}
      \resumeItem{Applied forward dynamics in real-time on motion-capture data using \textbf{MuJoCo}, providing joint-level control of the model and the option to extract physical data using inverse dynamics}
      \resumeItem{Explored the use of reinforcement learning to imitate input movements from motion capture suits in real-time}
      \resumeItemListEnd

      \resumeSubheading
      {\small{Research Assistant} \href{https://github.com/ff-zhang/t-cell-response-encoder/tree/master}{\footnotesize\faIcon{external-link-alt}}}{\small{May 2023 -- Present}}
      {\href{https://www.physics.utoronto.ca/~zilmana/}{Biological Physics Group}, University of Toronto}{}%{Python, PyTorch}
        \resumeItemListStart
        \resumeItem{Worked under Prof.\ Anton Zilman to model receptor signalling via soluble ligands with a high degree of cross-talk}
        \resumeItem{Built a feed-forward network in \textbf{PyTorch} that predicts the cytokine dynamics of T-cells in response to antigens}
        \resumeItem{Showed two variables are sufficient to determine cytokine concentrations (as predicted by previous theoretical work) because the model predicted the correct outputs with \emph{0.01\%} error using two hidden variables}
        \resumeItemListEnd
        
    % \item
    % \small\textbf{Identifying Suggestions in Student Evaluations using Deep Learning Models} \\
    % \hfill \textit{Python, TensorFlow, scikit-learn, Numpy}
      \resumeSubheading
      % {\small{Extracting Suggestions From Student Evaluations} \href{https://github.com/ff-zhang/extracting-suggestions}{\footnotesize\faIcon{external-link-alt}}}{\small{May 2022 -- Sept.\ 2022}}
      {\small{Research Student \href{https://github.com/ff-zhang/extracting-suggestions}{\footnotesize\faIcon{external-link-alt}}}}{\small{May 2022 -- Sept.\ 2022}}
      {Physics Education Group, University of Toronto}{}%{Python, TensorFlow, scikit-learn}
      \resumeItemListStart
        \resumeItem{Worked with Prof.\ Carolyn Sealfon to develop suggestion extraction models for feedback from physics courses}
        \resumeItem{Produced a dataset of \emph{{\raise.17ex\hbox{$\scriptstyle\mathtt{\sim}$}}11 000} sentences from student feedback which labels whether they contain suggestions}
        \resumeItem{Compared the effectiveness of statistical and deep-learning classifiers at extracting suggestions using \textbf{scikit-learn} and  \textbf{TensorFlow} respectively}
        % \resumeItem{Trained the models using \textbf{TensorFlow} and \textbf{scikit-learn} with a Bayesian hyperparameter optimizer}
        \resumeItem{Demonstrated the efficacy of a BERT classifier at addressing this problem with it achieving an F${}_1$ score of \emph{0.823}}
      \resumeItemListEnd
    \resumeSubHeadingListEnd

%-----------PROJECTS-----------
\section{Projects}
  \resumeSubHeadingListStart

    \resumeProjectHeading
    {\textbf{Student Response Classifier}}{\small{Mar.\ 2023 -- Apr.\ 2023}}
    \resumeItemListStart
      \resumeItem{Developed a 3-parameter logistic item response theory classifier in \textbf{PyTorch}, using alternating gradient descent for training, to predict the correctness of student answers to multiple-choice questions}
      \resumeItem{Obtained an accuracy of \emph{72\%} on the \textit{NeurIPS 2020 Education Challenge} dataset (within $5\%$ of the best solution)}
    \resumeItemListEnd

    \resumeProjectHeading
      {\textbf{MNIST Classifier} \href{https://github.com/ff-zhang/mnist-classifier}{\footnotesize\faIcon{external-link-alt}}}{\small{Dec.\ 2022 -- Jan.\ 2023}}%{\textit{C++, Eigen3}}
      \resumeItemListStart
        \resumeItem{Implemented the softmax classifier from \emph{Understanding Machine Learning -- from Theory to Algorithms} with stochastic gradient descent using \textbf{C++} and the linear algebra library \textbf{Eigen3}}
        \resumeItem{Achieved \emph{92\%} accuracy on the MNIST dataset of handwritten digits (within $2\%$ of the top classifier using SGD)}
        \resumeItem{Included the ability to save trained weights, perform batch training, and track the training error in real-time}
      \resumeItemListEnd

    \resumeSubHeadingListEnd

%-----------EXPERIENCE-----------
\section{Volunteer Activities}
% -----------Multiple Positions Heading-----------
%    \resumeSubSubheading
%     {Software Engineer I}{Oct 2014 - Sep 2016}
%     \resumeItemListStart
%        \resumeItem{Apache Beam}
%          {Apache Beam is a unified model for defining both batch and streaming data-parallel processing pipelines}
%     \resumeItemListEnd
%    \resumeSubHeadingListEnd
%-------------------------------------------

  \resumeSubHeadingListStart
    \resumeSubheading
    {Director of Internal Relations}{Apr.\ 2023 -- Present}{Computer Science Student Union, University of Toronto}{}
    \resumeItemListStart
      \resumeItem{Organized orientation for the {\raise.17ex\hbox{$\scriptstyle\mathtt{\sim}$}}500 undergraduate students entering the computer science (CMP1) stream}
      \resumeItem{Facilitated 10+ events in collaboration with various partners in industry (such as AMD and Google) and students organizations (such as UTMIST \href{https://utmist.gitlab.io/}{\footnotesize\faIcon{external-link-alt}} and WiCS \href{https://www.linkedin.com/company/uoftwics/}{\footnotesize\faIcon{external-link-alt}}) so far}
    \resumeItemListEnd

    \resumeSubHeadingListEnd

\end{document}