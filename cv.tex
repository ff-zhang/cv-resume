%-------------------------
% Resume in Latex
% Author : Jake Gutierrez
% Based off of: https://github.com/sb2nov/resume
% License : MIT
%------------------------

\documentclass[letterpaper,11pt]{article}

\usepackage{latexsym}
\usepackage[empty]{fullpage}
\usepackage{titlesec}
\usepackage{marvosym}
\usepackage[usenames,dvipsnames]{color}
\usepackage{verbatim}
\usepackage{enumitem}
\usepackage[hidelinks]{hyperref}
\usepackage{fancyhdr}
\usepackage[english]{babel}
\usepackage{tabularx}
\usepackage{fontawesome5}
\input{glyphtounicode}

\usepackage{amsmath, amsfonts, amsthm}
\usepackage{censor}

%----------FONT OPTIONS----------
% sans-serif
% \usepackage[sfdefault]{FiraSans}
% \usepackage[sfdefault]{roboto}
% \usepackage[sfdefault]{noto-sans}
% \usepackage[default]{sourcesanspro}

% serif
% \usepackage{CormorantGaramond}
% \usepackage{charter}


\pagestyle{fancy}
\fancyhf{} % clear all header and footer fields
\fancyfoot{}
\renewcommand{\headrulewidth}{0pt}
\renewcommand{\footrulewidth}{0pt}

% Adjust margins
\addtolength{\oddsidemargin}{-0.5in}
\addtolength{\evensidemargin}{-0.5in}
\addtolength{\textwidth}{1in}
\addtolength{\topmargin}{-.5in}
\addtolength{\textheight}{1.0in}

\urlstyle{same}

\raggedbottom
\raggedright
\setlength{\tabcolsep}{0in}

% Sections formatting
\titleformat{\section}{
  \vspace{-4pt}\scshape\raggedright\large
}{}{0em}{}[\color{black}\titlerule \vspace{-5pt}]

% Ensure that generate pdf is machine readable/ATS parsable
\pdfgentounicode=1

%-------------------------
% Custom commands
\newcommand{\resumeItem}[1]{
  \item\small{
    {#1 \vspace{-2pt}}
  }
}

\newcommand{\resumeSubheading}[5]{
  \vspace{-2pt}\item
    \begin{tabular*}{0.97\textwidth}[t]{l@{\extracolsep{\fill}}r}
      \small{\textbf{#1}; {#2}} & \small{#3} \\[-0.05em]
      \textit{\small{#4}} & \textit{\small #5}
    \end{tabular*}\vspace{-7pt}
}

\newcommand{\resumeSubheadingTwo}[3]{
  \vspace{-2pt}\item
    \begin{tabular*}{0.97\textwidth}[t]{l@{\extracolsep{\fill}}r}
      \small{\textbf{#1}; #2} & \small{#3}
    \end{tabular*}\vspace{-7pt}
}

\newcommand{\resumeSubheadingThree}[4]{
  \vspace{-2pt}\item
    \begin{tabular*}{0.97\textwidth}[t]{l@{\extracolsep{\fill}}r}
      \textbf{\small #1} & \small{#2} \\
      \textit{\small#3} & \textit{\small #4}
    \end{tabular*}\vspace{-7pt}
}

\newcommand{\resumeSubheadingFour}[2]{
  \vspace{-2pt}\item
    \begin{tabular*}{0.97\textwidth}[t]{l@{\extracolsep{\fill}}r}
      \textbf{\small #1} & \small{#2}
    \end{tabular*}\vspace{-7pt}
}

\newcommand{\resumeSubSubheading}[2]{
    \item
    \begin{tabular*}{0.97\textwidth}{l@{\extracolsep{\fill}}r}
      \textit{\small#1} & \textit{\small #2}
    \end{tabular*}\vspace{-7pt}
}

\newcommand{\resumeProjectHeading}[2]{
    \item
    \begin{tabular*}{0.97\textwidth}{l@{\extracolsep{\fill}}r}
      \small#1 & #2 \\
    \end{tabular*}\vspace{-7pt}
}

\newcommand{\resumeSubItem}[1]{\resumeItem{#1}\vspace{-4pt}}

\renewcommand\labelitemii{$\vcenter{\hbox{\tiny$\bullet$}}$}

\newcommand{\resumeSubHeadingListStart}{\begin{itemize}[leftmargin=0.15in, label={}]}
\newcommand{\resumeSubHeadingListEnd}{\end{itemize}}
\newcommand{\resumeItemListStart}{\begin{itemize}}
\newcommand{\resumeItemListEnd}{\end{itemize}\vspace{-5pt}}

\newcommand{\rom}[1]{\uppercase\expandafter{\romannumeral #1\relax}}

%-------------------------------------------

\begin{document}

\StopCensoring

%----------HEADING----------

\begin{center}
    \textbf{\LARGE \scshape Felix Zhang} \\ \vspace{1pt}
    \footnotesize\faIcon{envelope} \censor{\href{mailto:felixfzhang@cs.toronto.edu}{\underline{felixfzhang@cs.toronto.edu}}} $|$ 
    \footnotesize\faIcon{github} \small\censor{\href{https://github.com/ff-zhang}{\underline{ff-zhang}}} $|$ 
    \footnotesize\faIcon{linkedin} \small\censor{\href{https://www.linkedin.com/in/felixfzhang/}{\underline{felixfzhang}}}
    % \footnotesize\faIcon{globe} \censor{\href{https://ff-zhang.github.io/}{\underline{ff-zhang.github.io}}}
\end{center}


%-----------EDUCATION-----------
\section{Education}
\resumeSubHeadingListStart

\resumeSubheadingThree
  {University of Toronto}{Sept. 2025 -- June 2027}
  {Masters of Science in Computer Science}{}

\resumeSubheadingThree
  {University of Toronto}{Sept. 2021 -- June 2025}
  {Honours Bachelor of Science in Computer Science; Major in Mathematics}{\censor{3.96}/4.0 cGPA}

\resumeSubHeadingListEnd

\section{Publications}
\resumeSubHeadingListStart

\resumeSubheadingFour{PD3: Prefetching Data with DPUs for Disaggregated Memory}{May 2026}
\resumeSubItem{
  \vspace{-2pt}Sidharth Sankhe, Felix Zhang, Umayrah Chonee, Sherman Lim, Jason Hu, Jialin Li, Qizhen Zhang\\
  \textit{23\textsuperscript{\,rd} USENIX Symposium on Networked Systems Design and Implementation}
}

\resumeItemListEnd

%-----------EXPERIENCE-----------

\section{Research Experience}
\resumeSubHeadingListStart

\resumeSubheading
{Research Assistant}{\href{https://fardatalab.org/}{Far Data Lab}, University of Toronto}{\small{Sept.\ 2024 -- Present}}
{Supervisor:\ Prof.\ Qizhen Zhang}{}
  \resumeItemListStart
  \resumeItem{Investigated offloading computations onto SmartNICs and data processing unit (DPUs)}
  % \resumeItem{Implemented the \textit{Monodepth2} depth estimation model in \textbf{C++} from scratch}
  \resumeItem{Parallelized the execution of \textit{Monodepth2} in \textbf{Python} and \textbf{C++} with the latter achieving linear performance scaling with the number of threads}
  \resumeItem{Built a DPU-based prefetcher \textit{PD3} with a team of $6$ which intercepts network traffic to predict and prefetch data for tiered key-value stores}
  \resumeItem{Designed an external service for offloading shuffle operations from database management systems which supports either a memory or storage backend managed by DPUs}
  \resumeItemListEnd
  
\resumeSubheading
{Research Assistant}{University of Toronto \href{https://github.com/KidneyOS/KidneyOS}{\footnotesize\faIcon{external-link-alt}}}{\small{March 2024 -- Dec.\ 2024}}
{Supervisor:\ Prof.\ Jack Sun}{}
  \resumeItemListStart
  \resumeItem{Worked on a team of $11$ to implement a pedagogical kernel \textit{KidneyOS} in \textbf{Rust} to be used in an introductory operating systems course with \textbf{500+} students annually}
  \resumeItem{Enabled thread creation and destruction, multi-threading, pre-emptive scheduling within the thread system}
  \resumeItem{Led a team of $3$ to implement POSIX-compatible syscalls and add support running user-space executables}
  \resumeItemListEnd

\resumeSubheading
{Research Assistant}{\href{https://hollandbloorview.ca/research-education/bloorview-research-institute/research-centres-labs/prism-lab}{PRISM Lab}, Bloorview Research Institute}{June 2024 -- Aug.\ 2024}
{Supervisors:\ Erica Floreani and Prof.\ Tom Chau; Funded by:\ FUSRP}{}
  \resumeItemListStart
  % \resumeItem{Worked under  on denoising electroencephalogram (EEG) data with deep-learning models}
  \resumeItem{Curated deep-learning models from the literature on denoising electroencephalogram (EEG) data in a team of $4$ and benchmarked them on the \textit{EEGDenoiseNet} dataset}
  \resumeItem{Investigated the applicability of end-to-end transformer models to denoise EEG signals and the impact of using signals' time-frequency representation as input on model performance}
  \resumeItemListEnd

\resumeSubheading
{Research Assistant}{\href{https://www.physics.utoronto.ca/~zilmana/}{Biological Physics Group}, University of Toronto \href{https://github.com/ff-zhang/t-cell-response-encoder/tree/master}{\footnotesize\faIcon{external-link-alt}}}{May 2023 -- May 2024}
{Supervisor:\ Prof.\ Anton Zilman; Funded by:\ Work Study Program}{}
  \resumeItemListStart
  % \resumeItem{Worked with a group of $4$ to model receptor signalling via soluble ligands with a high degree of cross-talk}
  \resumeItem{Implemented a data pre-processing pipeline which processes raw cytokine data and extracts integral features}
  \resumeItem{Built a feed-forward network in \textbf{PyTorch} that predicts the cytokine dynamics of T cells in response to antigens}
  \resumeItem{On a team of $4$, showed two variables are sufficient to determine cytokine concentrations because our model predicted the correct output concentration with \textbf{0.01\%} error using a bottleneck layer with $2$ neurons}
  \resumeItemListEnd
    
\resumeSubheading
{Research Assistant}{Physics Education Group, University of Toronto \href{https://github.com/ff-zhang/extracting-suggestions}{\footnotesize\faIcon{external-link-alt}}}{May 2022 -- Sept.\ 2022}
{Supervisor: Prof.\ Carolyn Sealfon}{}
  \resumeItemListStart
  \resumeItem{Created a dataset of \textbf{{\raise.17ex\hbox{$\scriptstyle\mathtt{\sim}$}}11 000} sentences from student feedback which labels whether they contain suggestions}
  \resumeItem{Compared the effectiveness of statistical and deep-learning classifiers at identifying suggestions using \textbf{scikit-learn} and  \textbf{TensorFlow} respectively}
  \resumeItem{Demonstrated the efficacy of a BERT classifier at addressing this problem with it achieving an F${}_1$ score of \textbf{0.823}}
  \resumeItemListEnd

\resumeSubHeadingListEnd

\section{Industry Experience}
\resumeSubHeadingListStart

\resumeSubheadingTwo
{ML Runtime Engineer}{\href{https://www.cerebras.net/}{Cerebras Systems}}{May 2024 -- Present}
  \resumeItemListStart
  \resumeItem{Implemented a runtime virtual memory system in \textbf{C++} with a team of $3$ which pre-emptively loads data before it is accessed, allowing off-chip memory to be used for the first time with only a \textbf{10\%} performance penalty}
  \resumeItem{Added support for network storage in the paging system with remote direct memory access, providing \textbf{100} GB/s read and write speeds with \textbf{10} $\mu$s latency to multiple remote servers}
  \resumeItem{Enabled the ability log and replay the network operations, decreasing the time to recreate stalls and timeouts by over \textbf{80\%}, and setup unit tests to automatically catch breakages and performance regressions in the network layer}
  \resumeItem{Improved the throughput of the network layer by \textbf{6\%} when transferring data by implementing best practices for remote direct memory access and reducing setup overhead}
  \resumeItem{Determined the cable and port mapping for a one node cluster which will be used for all future deployments}
  \resumeItemListEnd

\resumeSubHeadingListEnd

%-----------AWARDS-----------
\section{Awards \& Scholarships}
  \resumeSubHeadingListStart
    \resumeProjectHeading
    {\textbf{Fields Undergraduate Summer Research Program} (\censor{\$3 800}), Fields Institute}{June -- Aug.\ 2024}
    \resumeProjectHeading
    {\textbf{Louis Savlov Scholarship in Sciences And Humanities} (\censor{\$1 000}), University of Toronto}{Nov.\ 2023 -- Jan.\ 2025}
    \resumeProjectHeading
    {\textbf{Dean's List Scholar}, University of Toronto}{Jan.\ 2022 -- June 2025}
    \resumeProjectHeading
    {\textbf{Second Malcom Wallace Scholarship} (\censor{\$5 000}), University of Toronto}{Sept.\ 2021 -- Oct.\ 2024}
    \resumeProjectHeading
    {\textbf{University of Toronto Scholar} (\censor{\$7 500}), University of Toronto}{Sept.\ 2021}
    \resumeProjectHeading
    {\textbf{B.C.\ Achievement Scholarship} (\censor{\$1 250}), Government of British Columbia}{Aug.\ 2021}
    \resumeProjectHeading
    {\textbf{District/Authority Scholarship} (\censor{\$1 250}), Government of British Columbia}{Aug.\ 2021}
  \resumeSubHeadingListEnd
  \vspace{-0.8em}

%-----------EXPERIENCE-----------
\section{Student Leadership}
  \resumeSubHeadingListStart

    \resumeSubheadingTwo
    {Director of Internal Relations}{Computer Science Student Union, University of Toronto}{Apr.\ 2023 -- Apr.\ 2024}
    \resumeItemListStart
      \resumeItem{Organized orientation for the \textbf{{\raise.17ex\hbox{$\scriptstyle\mathtt{\sim}$}}500} undergraduate students entering the computer science stream}
      \resumeItem{Planned \textbf{20+} events in collaboration with various partners in industry (such as AMD and Google) or student organizations (such as UTMIST \href{https://utmist.gitlab.io/}{\footnotesize\faIcon{external-link-alt}} and WiCS \href{https://www.linkedin.com/company/uoftwics/}{\footnotesize\faIcon{external-link-alt}})}
      \resumeItem{Hosted \textbf{5+} talks with professors in the Department of Computer Science at the University of Toronto}
    \resumeItemListEnd

    \resumeSubheadingTwo
    {First-Year Academic Officer}{Math Union, University of Toronto}{Sept.\ 2021 -- Apr.\ 2022}
    \resumeItemListStart
      \resumeItem{Facilitated discussions between \textbf{20} mentor-mentee pairs in the \textit{First-Year Mentorship Program} by providing guidance to the upper-year mentors}
      \resumeItem{Organized ``Coffee and Chat'' events which allowed for informal discussions between students and math professors}
    \resumeItemListEnd

    \resumeSubheadingTwo
    {Registered Study Group Leader}{Sidney Smith Commons, University of Toronto}{Sept. 2021 --April 2022}
    \resumeItemListStart
      \resumeItem{Led study groups for \textit{Foundations of Computer Science \rom{1}} and \textit{Enriched Introduction to the Theory of Computation}}
      \resumeItem{Headed weekly meetings for first-year students that reviewed content covered in the previous week's lecture}
      \resumeItem{Developed example problems to clarify and reinforce important concepts through group discussion}
    \resumeItemListEnd

    \resumeSubHeadingListEnd

%-----------PROJECTS-----------
\section{Projects}
  \resumeSubHeadingListStart

    \resumeProjectHeading
    {\textbf{Student Response Classifier}}{\small{Mar.\ 2023 -- Apr.\ 2023}}
    \resumeItemListStart
      \resumeItem{Developed a 3-parameter logistic item response theory classifier in \textbf{PyTorch}, using alternating gradient descent for training, to predict the correctness of student answers to multiple-choice questions}
      \resumeItem{Obtained an accuracy of \textbf{72\%} on the \textit{NeurIPS 2020 Education Challenge} dataset (within $5\%$ of the best solution)}
    \resumeItemListEnd

    \resumeProjectHeading
      {\textbf{Image Classifier} \href{https://github.com/ff-zhang/mnist-classifier}{\footnotesize\faIcon{external-link-alt}}}{\small{Dec.\ 2022 -- Jan.\ 2023}}
      \resumeItemListStart
        \resumeItem{Implemented a softmax classifier with stochastic gradient descent (SGD) from scratch in \textbf{C++} using only the linear algebra library \textbf{Eigen3}}
        \resumeItem{Achieved \textbf{92\%} accuracy on the \textit{MNIST} dataset of handwritten digits (within $2\%$ of the top classifier using SGD)}
        \resumeItem{Built in the ability to save trained weights, perform batch training, and track the training error in real-time}
      \resumeItemListEnd

    \resumeProjectHeading
    {\textbf{Image Restoration with Convolutional Neural Networks} \href{https://github.com/ff-zhang/research-code/blob/master/paper.pdf}{\footnotesize\faIcon{external-link-alt}}}{\small{Sept.\ 2020 -- June 2021}}
      \resumeItemListStart
      \resumeItem{Combined the models RIDNet and DeepDeblur using \textbf{PyTorch} to determine the ability of convolutional neural networks to deblur and denoise images}
      \resumeItem{Artificially generated a dataset of \textbf{5 000} noisy, blurred images using a Poisson-Gaussian noise model}
        \resumeItem{Discovered that integrating the two models offers marginal improvements over their individual performance}
      \resumeItemListEnd

%-----------PROGRAMMING SKILLS-----------
\section{Technical Skills}
  \begin{itemize}[leftmargin=0.15in, label={}]
      \item\begin{tabular}{l@{\hspace{0.8em}}l}
      \small\textbf{Languages} & \small{C++, Rust, Python, C, C\#, Bash, Java} \\
      \small\textbf{Frameworks} & \small{Catch2, PyTest, PyTorch, TensorFlow, OpenCV, scikit-learn, NumPy, Pandas, SciPy, Eigen3} \\
      \small\textbf{Tools} & \small{Git, Linux, Unix, tmux, CMake, Docker, Anaconda, Google Colab, QEMU, Jupyter, WSL, Slurm}
    \end{tabular}
  \end{itemize}

  \resumeSubHeadingListEnd

%-----------COURSEWORK-----------

\section{Selected Coursework}
\vspace{-0.3em}
\begin{minipage}{\textwidth}
  \centering
  \renewcommand*{\thefootnote}{\fnsymbol{footnote}}
  \renewcommand*{\thempfootnote}{\fnsymbol{mpfootnote}}
  \renewcommand{\arraystretch}{1.2}
  \begin{tabularx}{0.98\textwidth}{ p{6.0em} X r }
    \textbf{Code} & \textbf{Title} & \textbf{Term} \\
    \noalign{\vspace{0.1em}}
    \hline\hline
    \noalign{\vspace{0.2em}}
    CSC2306\footnotemark[1] & High Performance Scientific Computing & Winter 2025 \\
    CSC2525\footnotemark[1] & Research Topics in Database Management & Winter 2025 \\

    CSC2234\footnotemark[2] & Database System Technology & Fall 2025 \\
    CSC2235\footnotemark[1] & Cloud-native Data Management Systems & Fall 2025 \\

    CSC2221\footnotemark[1] & Introduction to the Theory of Distributed Computing & Fall 2024 \\

    CSC324 & Principles of Programming Languages & Winter 2024 \\
    CSC412\footnotemark[2] & Probabilistic Learning and Reasoning & Winter 2024 \\
    CSC413\footnotemark[2] & Neural Networks and Deep Learning & Winter 2024 \\
    CSC473 & Advanced Algorithm Design & Winter 2024 \\
    MAT357 & Real Analysis \rom{1} & Winter 2024 \\

    APM462 & Nonlinear Optimization & Fall 2023 \\
    CSC369 & Operating Systems & Fall 2023 \\
    CSC420 & Introduction to Image Understanding & Fall 2023 \\
    MAT354 & Complex Analysis \rom{1} & Fall 2023 \\
    MAT377 & Mathematical Probability & Fall 2023 \\

    MAT327 & Introduction to Topology & Summer 2023 \\

    CSC373 & Algorithm Design, Analysis and Complexity & Winter 2023 \\
    CSC384 & Introduction to Artificial Intelligence & Winter 2023 \\
    CSC438 & Computability and Logic & Winter 2023 \\

    CSC463 & Computational Complexity and Computability & Fall 2022 \\

    MAT344 & Introduction to Combinatorics & Summer 2022
  \end{tabularx}
  \footnotetext[1]{Graduate course}
  \footnotetext[2]{Cross-listed graduate course}
\end{minipage}

%-------------------------------------------

\end{document}