%-------------------------
% Resume in Latex
% Author : Jake Gutierrez
% Based off of: https://github.com/sb2nov/resume
% License : MIT
%------------------------

\documentclass[letterpaper,11pt]{article}

\usepackage{latexsym}
\usepackage[empty]{fullpage}
\usepackage{titlesec}
\usepackage{marvosym}
\usepackage[usenames,dvipsnames]{color}
\usepackage{verbatim}
\usepackage{enumitem}
\usepackage[hidelinks]{hyperref}
\usepackage{fancyhdr}
\usepackage[english]{babel}
\usepackage{tabularx}
\usepackage{fontawesome5}
\input{glyphtounicode}

\usepackage{amsmath, amsfonts, amsthm}
\usepackage{censor}

%----------FONT OPTIONS----------
% sans-serif
% \usepackage[sfdefault]{FiraSans}
% \usepackage[sfdefault]{roboto}
% \usepackage[sfdefault]{noto-sans}
% \usepackage[default]{sourcesanspro}

% serif
% \usepackage{CormorantGaramond}
% \usepackage{charter}


\pagestyle{fancy}
\fancyhf{} % clear all header and footer fields
\fancyfoot{}
\renewcommand{\headrulewidth}{0pt}
\renewcommand{\footrulewidth}{0pt}

% Adjust margins
\addtolength{\oddsidemargin}{-0.5in}
\addtolength{\evensidemargin}{-0.5in}
\addtolength{\textwidth}{1in}
\addtolength{\topmargin}{-.5in}
\addtolength{\textheight}{1.0in}

\urlstyle{same}

\raggedbottom
\raggedright
\setlength{\tabcolsep}{0in}

% Sections formatting
\titleformat{\section}{
  \vspace{-4pt}\scshape\raggedright\large
}{}{0em}{}[\color{black}\titlerule \vspace{-5pt}]

% Ensure that generate pdf is machine readable/ATS parsable
\pdfgentounicode=1

%-------------------------
% Custom commands
\newcommand{\resumeItem}[1]{
  \item\small{
    {#1 \vspace{-2pt}}
  }
}

\newcommand{\resumeSubheading}[5]{
  \vspace{-2pt}\item
    \begin{tabular*}{0.97\textwidth}[t]{l@{\extracolsep{\fill}}r}
      \small{\textbf{#1}; {#2}} & \small{#3} \\[-0.05em]
      \emph{\small{#4}} & \textit{\small #5} \\
    \end{tabular*}\vspace{-7pt}
}

\newcommand{\resumeSubheadingTwo}[3]{
  \vspace{-2pt}\item
    \begin{tabular*}{0.97\textwidth}[t]{l@{\extracolsep{\fill}}r}
      \small{\textbf{#1}; #2} & \small{#3}
    \end{tabular*}\vspace{-7pt}
}

\newcommand{\resumeSubheadingThree}[4]{
  \vspace{-2pt}\item
    \begin{tabular*}{0.97\textwidth}[t]{l@{\extracolsep{\fill}}r}
      \textbf{\small #1} & \small{#2} \\
      \textit{\small#3} & \textit{\small #4} \\
    \end{tabular*}\vspace{-7pt}
}

\newcommand{\resumeSubSubheading}[2]{
    \item
    \begin{tabular*}{0.97\textwidth}{l@{\extracolsep{\fill}}r}
      \textit{\small#1} & \textit{\small #2} \\
    \end{tabular*}\vspace{-7pt}
}

\newcommand{\resumeProjectHeading}[2]{
    \item
    \begin{tabular*}{0.97\textwidth}{l@{\extracolsep{\fill}}r}
      \small#1 & #2 \\
    \end{tabular*}\vspace{-7pt}
}

\newcommand{\resumeSubItem}[1]{\resumeItem{#1}\vspace{-4pt}}

\renewcommand\labelitemii{$\vcenter{\hbox{\tiny$\bullet$}}$}

\newcommand{\resumeSubHeadingListStart}{\begin{itemize}[leftmargin=0.15in, label={}]}
\newcommand{\resumeSubHeadingListEnd}{\end{itemize}}
\newcommand{\resumeItemListStart}{\begin{itemize}}
\newcommand{\resumeItemListEnd}{\end{itemize}\vspace{-5pt}}

\newcommand{\rom}[1]{\uppercase\expandafter{\romannumeral #1\relax}}

%-------------------------------------------
%%%%%%  RESUME STARTS HERE  %%%%%%%%%%%%%%%%%%%%%%%%%%%%

\begin{document}

\StopCensoring

%----------HEADING----------
% \begin{tabular*}{\textwidth}{l@{\extracolsep{\fill}}r}
%   \textbf{\href{http://sourabhbajaj.com/}{\Large Sourabh Bajaj}} & Email : \href{mailto:sourabh@sourabhbajaj.com}{sourabh@sourabhbajaj.com}\\
%   \href{http://sourabhbajaj.com/}{http://www.sourabhbajaj.com} & Mobile : +1-123-456-7890 \\
% \end{tabular*}

\begin{center}
    \textbf{\LARGE \scshape \censor{Felix Zhang}} \\ \vspace{1pt}
    \footnotesize\faIcon{linkedin} \small\censor{\href{https://www.linkedin.com/in/felixfzhang/}{\underline{felixfzhang}}} $|$ 
    \footnotesize\faIcon{github} \small\censor{\href{https://github.com/ff-zhang}{\underline{ff-zhang}}} $|$ 
    \footnotesize\faIcon{envelope} \censor{\href{mailto:felixf.zhang@utoronto.ca}{\underline{felixf.zhang@utoronto.ca}}}
    % \footnotesize\faIcon{phone} \censor{(778) 678-7918}
    % \footnotesize\faIcon{globe} \censor{\href{https://ff-zhang.github.io/}{\underline{ff-zhang.github.io}}}
\end{center}


%-----------EDUCATION-----------
\section{Education}
  \resumeSubHeadingListStart
    \resumeSubheadingThree
      {Honours Bachelor of Science, University of Toronto}{Sept. 2021 -- May 2025}
      {Specialist in Computer Science, Major in Mathematics}{\censor{3.95}/4.0 cGPA}
  \resumeSubHeadingListEnd

%-----------PROGRAMMING SKILLS-----------
\section{Technical Skills}
  \begin{itemize}[leftmargin=0.15in, label={}]
      \item\begin{tabular}{l@{\hspace{0.8em}}l}
      \small\textbf{Languages} & \small{Python, C, C++, Rust, Java} \\
      \small\textbf{Frameworks} & \small{PyTorch, TensorFlow, scikit-learn, NumPy, Pandas, SciPy, Matplotlib, MuJoCo, Eigen3} \\ % Vue.js, Django, Bootstrap4, Redis, MySQL, spaCy
      \small\textbf{Tools} & \small{Git, shell, ssh, Unix, CMake, Anaconda, Google Colab, QEMU, Jupyter, WSL, Slurm}
    \end{tabular}
  \end{itemize}

%-----------EXPERIENCE-----------
  \section{Experience}
  \resumeSubHeadingListStart
  
  \resumeSubheading
  {Research Assistant}{\href{https://hollandbloorview.ca/research-education/bloorview-research-institute/research-centres-labs/prism-lab}{PRISM Lab}, Bloorview Research Institute}{June 2024 -- Present}
  {Supervisors: Erica Floreani and Prof.\ Tom Chau; Funded by: FUSRP}{}
    \resumeItemListStart
    % \resumeItem{Worked under  on denoising electroencephalogram (EEG) data with deep-learning models}
    \resumeItem{Curated deep-learning models from the literature on denoising electroencephalogram (EEG) data in a team of $4$ and benchmarked them on the \texttt{EEGDenoiseNet} dataset}
    \resumeItem{Investigated the applicability of end-to-end transformer models to denoise EEG signals and the impact of using signals' time-frequency representation as input}
    \resumeItemListEnd
    
  \resumeSubheadingTwo
  {ML Runtime Engineer}{\href{https://www.cerebras.net/}{Cerebras Systems}}{May 2024 -- Present}
    \resumeItemListStart
    \resumeItem{Implemented a runtime virtual memory system in \textbf{C++} which pre-emptively loads data immediately before it is accessed, enabling off-chip memory to be used for the first time with only a \textbf{\%10} performance penalty}
    \resumeItem{Triaged and addressed \textbf{20+} failing tests within a high-priority test suite which is run nightly}
    \resumeItem{Developed a test variant which enabled performance testing of a single runtime node for the first time and decreased testing overhead by avoiding the generation of unnecessary input tensors}
    \resumeItemListEnd
    
  \resumeSubheading
  {Research Assistant}{University of Toronto \href{https://github.com/KidneyOS/KidneyOS}{\footnotesize\faIcon{external-link-alt}}}{\small{May 2024 -- Present}}
  {Supervisor: Prof.\ Jack Sun}{}
    \resumeItemListStart
    \resumeItem{Worked on a team of $11$ to implement a pedagogical kernel \emph{KidneyOS} in \textbf{Rust} to be used in an introductory operating systems course with \textbf{500+} students annually}
    \resumeItem{Built the process control system and integrated the user thread handling into our context switching module}
    \resumeItem{Implemented kernel/user thread differentiation and the \texttt{exit}, \texttt{fork}, \texttt{read}, \texttt{write}, and \texttt{waitpid} system calls}
    \resumeItemListEnd

  \resumeSubheading
  {Machine Learning Researcher}{\href{https://bmolab.artsci.utoronto.ca/}{BMO Lab}, University of Toronto \href{https://github.com/ff-zhang/mocap-mujoco}{\footnotesize\faIcon{external-link-alt}}}{July 2023 -- May 2024}
  {Supervisor: Prof.\ David Rokeby}{}
    \resumeItemListStart
    % \resumeItem{Worked under  to integrate motion-capture suits and diffusion models into live performances}
    \resumeItem{Applied forward dynamics in real-time on motion-capture data using \textbf{MuJoCo}, providing joint-level control of the model and the option to extract physical data using inverse dynamics}
    \resumeItem{Used imitation learning to enable humanoid models to copy movements from motion capture suits in real-time}
    \resumeItemListEnd

    \resumeSubheading
    {Research Assistant}{\href{https://www.physics.utoronto.ca/~zilmana/}{Biological Physics Group}, University of Toronto \href{https://github.com/ff-zhang/t-cell-response-encoder/tree/master}{\footnotesize\faIcon{external-link-alt}}}{May 2023 -- May 2024}
    {Supervisor: Prof.\ Anton Zilman}{}
      \resumeItemListStart
      % \resumeItem{Worked with a group of $4$ to model receptor signalling via soluble ligands with a high degree of cross-talk}
      \resumeItem{Reimplemented a data pre-processing pipeline which processes raw cytokine data and extracts integral features}
      \resumeItem{Built a feed-forward network in \textbf{PyTorch} that predicts the cytokine dynamics of T cells in response to antigens}
      \resumeItem{With a team of $4$, showed two variables are sufficient to determine cytokine concentrations because our model predicted the correct output concentration with \textbf{0.01\%} error using a bottleneck with $2$ dimensions}
      \resumeItemListEnd
      
    \resumeSubheading
    {Research Student}{Physics Education Group, University of Toronto \href{https://github.com/ff-zhang/extracting-suggestions}{\footnotesize\faIcon{external-link-alt}}}{May 2022 -- Sept.\ 2022}
    {Supervisor: Prof.\ Carolyn Sealfon}{}
    \resumeItemListStart
      % \resumeItem{Worked with  to develop suggestion extraction models for feedback from physics courses}
      \resumeItem{Created a dataset of \textbf{{\raise.17ex\hbox{$\scriptstyle\mathtt{\sim}$}}11 000} sentences from student feedback which labels whether they contain suggestions}
      \resumeItem{Compared the effectiveness of statistical and deep-learning classifiers at identifying suggestions using \textbf{scikit-learn} and  \textbf{TensorFlow} respectively}
      \resumeItem{Demonstrated the efficacy of a BERT classifier at addressing this problem with it achieving an F${}_1$ score of \textbf{0.823}}
    \resumeItemListEnd
  \resumeSubHeadingListEnd

%-----------PROJECTS-----------
\section{Projects}
  \resumeSubHeadingListStart

    \resumeProjectHeading
    {\textbf{Image Domain Adaption} \href{https://github.com/ff-zhang/domain-adaption-optimal-transport}{\footnotesize\faIcon{external-link-alt}}}{\small{Sept.\ 2023 -- Dec.\ 2023}}
    \resumeItemListStart
      \resumeItem{Worked with \href{https://www.math.toronto.edu/joaqsan/JoaquinPersonal.html}{Joaquin Sanchez-Garcia} on applying the theory of optimal transport to domain adaption within the context of image classification.}
      \resumeItem{Used \textbf{Python Optimal Transport} to compute various functions which transform the \texttt{EMNIST} dataset of handwritten digits such that its distribution and priors match those of the \texttt{MNIST} dataset.}
      \resumeItem{Found that the accuracy of a fully-connected feed-forward classifier trained on the \texttt{MNIST} dataset was improved from \emph{17\%} to \emph{73\%} on the \texttt{EMNIST} dataset, demonstrating its usefulness.}
    \resumeItemListEnd

    \resumeProjectHeading
    {\textbf{Student Response Classifier}}{\small{Mar.\ 2023 -- Apr.\ 2023}}
    \resumeItemListStart
      \resumeItem{Developed a 3-parameter logistic item response theory classifier in \textbf{PyTorch}, using alternating gradient descent for training, to predict the correctness of student answers to multiple-choice questions}
      \resumeItem{Obtained an accuracy of \textbf{72\%} on the \textit{NeurIPS 2020 Education Challenge} dataset (within $5\%$ of the best solution)}
    \resumeItemListEnd

    \resumeProjectHeading
      {\textbf{MNIST Classifier} \href{https://github.com/ff-zhang/mnist-classifier}{\footnotesize\faIcon{external-link-alt}}}{\small{Dec.\ 2022 -- Jan.\ 2023}}%{\textit{C++, Eigen3}}
      \resumeItemListStart
        \resumeItem{Implemented a softmax classifier with stochastic gradient descent (SGD) from scratch in \textbf{C++} using only the linear algebra library \textbf{Eigen3}}
        \resumeItem{Achieved \textbf{92\%} accuracy on the \texttt{MNIST} dataset of handwritten digits (within $2\%$ of the top classifier using SGD)}
        \resumeItem{Built in the ability to save trained weights, perform batch training, and track the training error in real-time}
      \resumeItemListEnd

      \resumeProjectHeading
      {\textbf{Image Restoration with Convolutional Neural Networks} \href{https://github.com/ff-zhang/research-code/blob/master/paper.pdf}{\footnotesize\faIcon{external-link-alt}}}{\small{Sept.\ 2020 -- June 2021}}
        \resumeItemListStart
        \resumeItem{Combined the models RIDNet and DeepDeblur using \textbf{PyTorch} to determine the ability of convolutional neural networks to deblur and denoise images}
        \resumeItem{Artificially generated a dataset of \textbf{{\raise.17ex\hbox{$\scriptstyle\mathtt{\sim}$}}5 000} noisy, blurred images using a Poisson-Gaussian noise model}
          \resumeItem{Discovered that integrating the two models offers marginal improvements over their individual performance}
        \resumeItemListEnd

    \resumeSubHeadingListEnd

%-----------EXPERIENCE-----------
\section{Student Leadership}

  \resumeSubHeadingListStart
    \resumeSubheadingTwo
    {Director of Internal Relations}{Computer Science Student Union, University of Toronto}{Apr.\ 2023 -- Apr.\ 2024}
    \resumeItemListStart
      \resumeItem{Organized orientation for the \textbf{{\raise.17ex\hbox{$\scriptstyle\mathtt{\sim}$}}500} undergraduate students entering the computer science stream}
      \resumeItem{Planned \textbf{20+} events in collaboration with various partners in industry (such as AMD and Google) and student organizations (such as UTMIST \href{https://utmist.gitlab.io/}{\footnotesize\faIcon{external-link-alt}} and WiCS \href{https://www.linkedin.com/company/uoftwics/}{\footnotesize\faIcon{external-link-alt}})}
      \resumeItem{Hosted \textbf{5+} talks with professors in the Department of Computer Science at the University of Toronto}
    \resumeItemListEnd

    \resumeSubheadingTwo
    {First-Year Academic Officer}{Math Union, University of Toronto}{Sept.\ 2021 -- Apr.\ 2022}
    \resumeItemListStart
      \resumeItem{Facilitated discussions between \textbf{20} mentor-mentee pairs in the \textit{First-Year Mentorship Program} by providing guidance to the upper-year mentors}
      \resumeItem{Organized ``Coffee and Chat'' events which allowed for informal discussions between students and math professors}
    \resumeItemListEnd

    \resumeSubheadingTwo
    {Registered Study Group Leader}{Sidney Smith Commons, University of Toronto}{Sept. 2021 -- April 2022}
    \resumeItemListStart
      \resumeItem{Led study groups for \textit{Foundations of Computer Science \rom{1}} and \textit{Enriched Introduction to the Theory of Computation}}
      \resumeItem{Headed weekly meetings for first-years students that reviewed content covered in the previous week's lecture}
      \resumeItem{Developed example problems to clarify and reinforce important concepts through group discussion}
    \resumeItemListEnd

    \resumeSubHeadingListEnd

    %-----------AWARDS-----------
    \section{Awards \& Scholarships}
    
      \resumeSubHeadingListStart
        \resumeProjectHeading
        {\textbf{Fields Undergraduate Summer Research Program} (FUSRP), University of Toronto}{June 2024 -- Present}
        \resumeProjectHeading
        {\textbf{Louis Savlov Scholarships in Sciences And Humanities} (\$500), University of Toronto}{Nov.\ 2023}
        \resumeProjectHeading
        {\textbf{Dean's List Scholar}, University of Toronto}{Sept.\ 2021 -- Present}
        \resumeProjectHeading
        {\textbf{Second Malcom Wallace Scholarship} (\$4 500), University of Toronto}{Sept.\ 2021 -- Present}
        \resumeProjectHeading
        {\textbf{University of Toronto Scholar} (\$7 500), University of Toronto}{Sept.\ 2021}
        \resumeProjectHeading
        {\textbf{B.C.\ Achievement Scholarship} (\$1 250), Government of British Columbia}{Aug.\ 2021}
        \resumeProjectHeading
        {\textbf{District/Authority Scholarship} (\$1 250), Government of British Columbia}{Aug.\ 2021}
      \resumeSubHeadingListEnd
      \vspace{-0.8em}

    %-----------COURSEWORK-----------

    \section{Selected Coursework}
    \vspace{-0.3em}
    \begin{minipage}{\textwidth}
      \centering
      \renewcommand*{\thefootnote}{\fnsymbol{footnote}}
      \renewcommand*{\thempfootnote}{\fnsymbol{mpfootnote}}
      \renewcommand{\arraystretch}{1.2}
      \begin{tabularx}{0.98\textwidth}{ p{6.0em} X r }
        \textbf{Code} & \textbf{Title} & \textbf{Term} \\
        \noalign{\vspace{0.1em}}
        \hline\hline
        \noalign{\vspace{0.2em}}
        CSC324 & Principles of Programming Languages & Winter 2024 \\
        CSC412\footnotemark[2] & Probabilistic Learning and Reasoning & Winter 2024 \\
        CSC413\footnotemark[2] & Neural Networks and Deep Learning & Winter 2024 \\
        CSC473 & Advanced Algorithm Design & Winter 2024 \\
        MAT357 & Real Analysis \rom{1} & Winter 2024 \\

        % MAT347 & Groups, Rings, and Fields & Fall 2023/Winter 2024 \\

        APM462 & Nonlinear Optimization & Fall 2023 \\
        CSC369 & Operating Systems & Fall 2023 \\
        CSC420 & Introduction to Image Understanding & Fall 2023 \\
        MAT354 & Complex Analysis \rom{1} & Fall 2023 \\
        MAT377 & Mathematical Probability & Fall 2023 \\

        % MAT327 & Introduction to Topology & Summer 2023 \\

        % CSC311 & Introduction to Machine Learning & Winter 2023 \\
        CSC373 & Algorithm Design, Analysis and Complexity & Winter 2023 \\
        CSC384 & Introduction to Artificial Intelligence & Winter 2023 \\
        % CSC438 & Computability and Logic & Winter 2023 \\

        % MAT257 & Analysis \rom{2} & Fall 2022/Winter 2023 \\

        % CSC258 & Computer Organization & Fall 2022 \\
        % CSC265 & Enriched Data Structures and Analysis & Fall 2022 \\
        CSC463 & Computational Complexity and Computability & Fall 2022 \\

        % MAT344 & Introduction to Combinatorics & Summer 2022
      \end{tabularx}
      \footnotetext[2]{Cross-listed graduate courses}
    \end{minipage}

%-------------------------------------------

\end{document}